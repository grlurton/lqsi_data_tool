%% Generated by Sphinx.
\def\sphinxdocclass{report}
\documentclass[letterpaper,10pt,english]{sphinxmanual}
\ifdefined\pdfpxdimen
   \let\sphinxpxdimen\pdfpxdimen\else\newdimen\sphinxpxdimen
\fi \sphinxpxdimen=.75bp\relax

\usepackage[utf8]{inputenc}
\ifdefined\DeclareUnicodeCharacter
 \ifdefined\DeclareUnicodeCharacterAsOptional
  \DeclareUnicodeCharacter{"00A0}{\nobreakspace}
  \DeclareUnicodeCharacter{"2500}{\sphinxunichar{2500}}
  \DeclareUnicodeCharacter{"2502}{\sphinxunichar{2502}}
  \DeclareUnicodeCharacter{"2514}{\sphinxunichar{2514}}
  \DeclareUnicodeCharacter{"251C}{\sphinxunichar{251C}}
  \DeclareUnicodeCharacter{"2572}{\textbackslash}
 \else
  \DeclareUnicodeCharacter{00A0}{\nobreakspace}
  \DeclareUnicodeCharacter{2500}{\sphinxunichar{2500}}
  \DeclareUnicodeCharacter{2502}{\sphinxunichar{2502}}
  \DeclareUnicodeCharacter{2514}{\sphinxunichar{2514}}
  \DeclareUnicodeCharacter{251C}{\sphinxunichar{251C}}
  \DeclareUnicodeCharacter{2572}{\textbackslash}
 \fi
\fi
\usepackage{cmap}
\usepackage[T1]{fontenc}
\usepackage{amsmath,amssymb,amstext}
\usepackage{babel}
\usepackage{times}
\usepackage[Bjarne]{fncychap}
\usepackage[dontkeepoldnames]{sphinx}

\usepackage{geometry}

% Include hyperref last.
\usepackage{hyperref}
% Fix anchor placement for figures with captions.
\usepackage{hypcap}% it must be loaded after hyperref.
% Set up styles of URL: it should be placed after hyperref.
\urlstyle{same}

\addto\captionsenglish{\renewcommand{\figurename}{Fig.}}
\addto\captionsenglish{\renewcommand{\tablename}{Table}}
\addto\captionsenglish{\renewcommand{\literalblockname}{Listing}}

\addto\captionsenglish{\renewcommand{\literalblockcontinuedname}{continued from previous page}}
\addto\captionsenglish{\renewcommand{\literalblockcontinuesname}{continues on next page}}

\addto\extrasenglish{\def\pageautorefname{page}}

\setcounter{tocdepth}{1}



\title{XLQSI Documentation}
\date{Jan 28, 2018}
\release{}
\author{Grégoire Lurton}
\newcommand{\sphinxlogo}{\vbox{}}
\renewcommand{\releasename}{Release}
\makeindex

\begin{document}

\maketitle
\sphinxtableofcontents
\phantomsection\label{\detokenize{index::doc}}



\chapter{General Presentation}
\label{\detokenize{intro::doc}}\label{\detokenize{intro:general-presentation}}\label{\detokenize{intro:documentation-for-xlqsi-s}}
The need for an Excel tool to collect data related to the continuous quality strengthening of laboratories was first raised by the Laboratory Systems
Strengthening team in ITech. In order to use the Laboratory Quality Stepwise Implementation (LQSI) tool, developed by the World Health Organization in collaboration with the Dutch Royal Tropical Institute (KIT), field teams needed a tool that would be easy to use, adaptable to specific conditions or local regulations in the field, and that would allow a better follow-up of laboratories progresses and results.

The wide availability of Excel on computers around the world, and the increasing familiarity of most health workers with this software made it an easy choice for a technology in which to build a new tool. A first round of transcription of LQSI in Excel in Cambodia proved successful in offering an easy to use, widely understood implementation. In the summer of 2017, the decision was made to adapt this first version to make it an adaptable tool that could be adapted in different settings, and that would be well suited for advanced data analysis.


\section{From LQSI to XLQSI}
\label{\detokenize{intro:from-lqsi-to-xlqsi}}
The LQSI is comprehensive framework developed by WHO and KIT to help laboratory managers to implement quality improvement programs. The framework is presented \sphinxhref{https://extranet.who.int/lqsi/}{on WHO’s website} which allows for an easy navigation and comprehensive explanations.

LQSI is a four phases process towards laboratory quality improvement. In each phase, a set of essential activities are defined, that should be carried out. These activities are classified in 12 categories. Additionally, for each activity, a checklist is defined to ensure that is has been well implemented. The WHO’s website conveniently provides different ways to generate checklists by phase or by category of activity, and roadmaps to follow the relationships and successions of each activity.

Meanwhile, when implementing this framework, two main problems can be encountered:
1. The high dimensionality of the framework and the multiplicity of items can make hard to navigate and get an overview of.
2. The checklists defined are generic, and may be hard to use in any specific setting.

To palliate this problem, XLQSI has been designed to allow users to easily \sphinxstylestrong{navigate and visualize} activities categories, and to have \sphinxstylestrong{synthetic visualizations} of the activities carried and their results. It also contains different \sphinxstylestrong{completion checks} and \sphinxstylestrong{progression bars} to allow users to manage their work time efficiently. Finally, XLQSI offers a framework to \sphinxstylestrong{adapt completion checks} for the different activities to local contexts, thus helping improving the usability of the tool. Refer to \sphinxhref{create\_data\_entry.html\#define-the-completion-checks}{the appropriate paragraph} for an explanation of how to define local checklists.

\begin{sphinxadmonition}{important}{Important:}
An important terminology note: we differentiate between the \sphinxstyleemphasis{checklist items}, which are the generic checks offered in LQSI, and the \sphinxstyleemphasis{completions checks}, which are the context specific checks defined by the users when they set-up data collection.
\end{sphinxadmonition}


\section{XLQSI overview}
\label{\detokenize{intro:xlqsi-overview}}
XLQSI is made of two main modules. All the modules can be \sphinxhref{https://github.com/grlurton/xlqsi/blob/master/xlqsi.zip}{downloaded from the github repository}. The current version of XLQSI is developed in Excel enriched in Python using the \sphinxhref{https://www.xlwings.org/}{xlwings library and add-in}. The functions packaged in the downloaded zip file do not necessitate any further installation, but only on Windows. Please refer to to {\hyperref[\detokenize{technical::doc}]{\sphinxcrossref{\DUrole{doc}{technical section}}}} or on the github repository to learn how to access and modify non packaged versions.

\begin{sphinxadmonition}{warning}{Warning:}
While setting-up the data entry module and importing data for visualization, XLQSI needs to access the packaged Python libraries. For this reason, it is important that the original set-up and the data importation be made in an arborescence that respects the file organization in the downloaded zip file.
\end{sphinxadmonition}


\subsection{Data Entry Module}
\label{\detokenize{intro:data-entry-module}}
The data entry module is the main tool for field workers to collect data on their laboratory quality activities. It can be adapted to a specific setting before being used. The different steps for the initial setup of the data entry is described in the {\hyperref[\detokenize{create_data_entry::doc}]{\sphinxcrossref{\DUrole{doc}{data entry set-up}}}} page.

Once a data entry template is defined, it can be shared and used in different labs. The data is entered in two different steps, which are described in the {\hyperref[\detokenize{fill_data_entry::doc}]{\sphinxcrossref{\DUrole{doc}{data entry}}}} page. At this stage, the user can already have an overview of the current results and performance of the facility.


\subsection{Data Visualization Module}
\label{\detokenize{intro:data-visualization-module}}
The data visualization module is a separate Excel workbook, in which the user can compile all the different data collection iterations for a given laboratory, and visualize the evolutions of the performance, and the strong and weak points of this facility. The use of this module is described in the {\hyperref[\detokenize{data_viz::doc}]{\sphinxcrossref{\DUrole{doc}{Data Visualization}}}} page.


\chapter{General Introduction}
\label{\detokenize{general::doc}}\label{\detokenize{general:general-introduction}}
\begin{sphinxadmonition}{note}{\label{general:index-0}Todo:}
Description of different tabs / sheets
\end{sphinxadmonition}

\begin{sphinxadmonition}{note}{\label{general:index-1}Todo:}
General workflow and use cases
\end{sphinxadmonition}


\chapter{Setting-up a Data entry file}
\label{\detokenize{create_data_entry::doc}}\label{\detokenize{create_data_entry:setting-up-a-data-entry-file}}

\section{Downloading and unpacking}
\label{\detokenize{create_data_entry:downloading-and-unpacking}}\begin{enumerate}
\item {} 
Download the most recent version of the files should be downloaded on \sphinxurl{https://github.com/grlurton/xlqsi/blob/master/xlqsi.zip}. The zip file contains the two modules, as well as the Python libraries necessary to run the different contents.

\item {} 
Copy the zipfile in the directory you want to use for your project.

\item {} 
Unzip the zipfile.

\end{enumerate}

\begin{sphinxadmonition}{warning}{Warning:}
For all the data entry setup, it is important keep the data entry file in the same directory as the \sphinxstyleemphasis{build} folder, so that it can access the necessary libraries.
\end{sphinxadmonition}

Once  you open the Excel files, you will need to allow macros to be able to use some of the functions.

\begin{figure}[htbp]
\centering
\capstart

\noindent\sphinxincludegraphics{{security_check}.png}
\caption{Allowing macros (french translation)}\label{\detokenize{create_data_entry:security-check}}\end{figure}


\section{Set language}
\label{\detokenize{create_data_entry:set-language}}
It is currently possible to use XLQSI in five different languages. The English, French, Russian and Spanish versions have been directly taken from the WHO’s website. The Khmer translation was made by ITech’s team in Cambodia.

Language selection happens on the first tab of the data entry workbook, just by choosing the desired language in the dropdown menu as shown in \hyperref[\detokenize{create_data_entry:translation}]{Fig.\@ \ref{\detokenize{create_data_entry:translation}}}.

\begin{figure}[htbp]
\centering
\capstart

\noindent\sphinxincludegraphics{{translation_image}.png}
\caption{Setting the translation}\label{\detokenize{create_data_entry:translation}}\end{figure}

Adding a new translation language to XLQSI is a simple even if very systematic task. The procedure to adapt translations or add new ones is described in {\hyperref[\detokenize{technical::doc}]{\sphinxcrossref{\DUrole{doc}{technical documentation}}}}.


\section{Define the completion checks}
\label{\detokenize{create_data_entry:define-the-completion-checks}}\label{\detokenize{create_data_entry:id1}}
To get more background on the difference between the \sphinxstyleemphasis{checklist items} and the \sphinxstyleemphasis{completion checks} please refer to the {\hyperref[\detokenize{intro::doc}]{\sphinxcrossref{\DUrole{doc}{introductory content}}}}.

Completion checks are defined by the users to adapt generic checklist items to a specific context. For example, one of the checklist items defined by WHO are “Are sharps securely and safely incinerated without the risk for needle accidents?”. In Cambodia, for this item to be validated, it necessitated the completion of three completion checks :
1. All sharps wastes are stored in sharps container,
2. Sharps containers are autoclaved and
3. Sharps containers are incinerated properly

We encourage each user to think on how much they need to adapt completion checks to their own context. For users using XLQSI in the context of the implementation of national policies, they should refer to national norms and procedures to accurately define completion checks. Additionally, different completion checks may have to be defined for different types of laboratories. For examples of how completion checks can be defined, we share the example of how ITech has been defining them in \sphinxhref{https://github.com/grlurton/xlqsi/tree/master/examples}{Cambodia in XLSQI’s repository examples}.

\begin{sphinxadmonition}{note}{Note:}
It is not necessary to define completion checks for all checklist items. For checklist items that are directly applicable to the user’s situation, the user will be able to validate the item directly during data entry.
\end{sphinxadmonition}

The definition of the completion checks is made in two steps:
\begin{enumerate}
\item {} 
In each \sphinxstyleemphasis{Criterias} sheet (\sphinxstyleemphasis{Phase1 - Criterias}, \sphinxstyleemphasis{Phase2 - Criterias}, \sphinxstyleemphasis{Phase3 - Criterias} and \sphinxstyleemphasis{Phase4 - Criterias}), in the \sphinxstyleemphasis{Completion Check} columns, the user can define the completion checks he needs. In \hyperref[\detokenize{create_data_entry:define-completion-check}]{Fig.\@ \ref{\detokenize{create_data_entry:define-completion-check}}}, we can see how the checklist item \sphinxstyleemphasis{Have critical environmental/equipment parameters been identified?} was adapted into four completion checks.

\end{enumerate}

\begin{figure}[htbp]
\centering
\capstart

\noindent\sphinxincludegraphics{{define_completion_check}.png}
\caption{Examples of completion checks for a given checklist item}\label{\detokenize{create_data_entry:define-completion-check}}\end{figure}

Also, you can access each of the sheet through the links in the dedicated box in the  \sphinxstyleemphasis{Getting Started} sheet (see \hyperref[\detokenize{create_data_entry:click-completion}]{Fig.\@ \ref{\detokenize{create_data_entry:click-completion}}}).
\begin{enumerate}
\setcounter{enumi}{1}
\item {} 
The goal of XLQSI is to make data collection easier. Once all the completion checks have been defined for each phase, you can thus import them in synthetic lists for each phase. This is done by clicking on the \sphinxstyleemphasis{Extract Criterias Checklist} in the \sphinxstyleemphasis{Getting Started} sheet. On the right side of \hyperref[\detokenize{create_data_entry:click-completion}]{Fig.\@ \ref{\detokenize{create_data_entry:click-completion}}}, you can see a counter. This count shows how many completion checks have been imported. Wait until the counter stops going up before modifying completion checks.

\end{enumerate}

\begin{figure}[htbp]
\centering
\capstart

\noindent\sphinxincludegraphics{{import_checklist}.png}
\caption{Completion checks importation}\label{\detokenize{create_data_entry:click-completion}}\end{figure}

The completion of these lists is explained in the section on {\hyperref[\detokenize{fill_data_entry::doc}]{\sphinxcrossref{\DUrole{doc}{data entry}}}}.


\section{Freezing the data entry file}
\label{\detokenize{create_data_entry:freezing-the-data-entry-file}}
It may be useful to freeze the data entry file, to prevent undue modification by users, or to make its use easier. In the current stage of XLQSI, there is no systematic way to do it, and we welcome feedback of users regarding which aspect of the file could be deleted or frozen (this can be communicated through the \sphinxhref{https://github.com/grlurton/xlqsi/issues/2}{dedicated github issue} ).


\chapter{Collecting data}
\label{\detokenize{fill_data_entry::doc}}\label{\detokenize{fill_data_entry:collecting-data}}

\section{Importing Equipment List}
\label{\detokenize{fill_data_entry:importing-equipment-list}}

\section{Filling in completion checks}
\label{\detokenize{fill_data_entry:filling-in-completion-checks}}

\section{Validation checklist items}
\label{\detokenize{fill_data_entry:validation-checklist-items}}

\chapter{Data vizualisation}
\label{\detokenize{data_viz::doc}}\label{\detokenize{data_viz:data-vizualisation}}
In order


\chapter{Technical Description}
\label{\detokenize{technical::doc}}\label{\detokenize{technical:technical-description}}
\begin{sphinxadmonition}{note}{\label{technical:index-0}Todo:}
Include explanation of development and compiling of XLQSI
\end{sphinxadmonition}


\chapter{Helping improving XLQSI}
\label{\detokenize{contribute::doc}}\label{\detokenize{contribute:helping-improving-xlqsi}}
A PDF version of this documentation can be downloaded \sphinxhref{https://github.com/grlurton/xlqsi/raw/master/docs/latex/XLQSI.pdf}{here}.



\renewcommand{\indexname}{Index}
\printindex
\end{document}